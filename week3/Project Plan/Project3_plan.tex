% --------------------------------------------------------------
% This is all preamble stuff that you don't have to worry about.
% Head down to where it says "Start here"
% --------------------------------------------------------------

\documentclass[11pt]{article}

\usepackage[margin=1in]{geometry}
\usepackage{float,amsmath,amsthm,amssymb,caption}
\usepackage{algorithmic, algorithm}
\newcommand{\N}{\mathbb{N}}
\newcommand{\Z}{\mathbb{Z}}

\newenvironment{theorem}[2][Theorem]{\begin{trivlist}
\item[\hskip \labelsep {\bfseries #1}\hskip \labelsep {\bfseries #2.}]}{\end{trivlist}}
\newenvironment{lemma}[2][Lemma]{\begin{trivlist}
\item[\hskip \labelsep {\bfseries #1}\hskip \labelsep {\bfseries #2.}]}{\end{trivlist}}
\newenvironment{exercise}[2]{\begin{trivlist}
\item[\hskip \labelsep {\bfseries #1}\hskip \labelsep {\bfseries #2.}]}{\end{trivlist}}
\newenvironment{reflection}[2][Reflection]{\begin{trivlist}
\item[\hskip \labelsep {\bfseries #1}\hskip \labelsep {\bfseries #2.}]}{\end{trivlist}}
\newenvironment{proposition}[2][Proposition]{\begin{trivlist}
\item[\hskip \labelsep {\bfseries #1}\hskip \labelsep {\bfseries #2.}]}{\end{trivlist}}
\newenvironment{corollary}[2][Corollary]{\begin{trivlist}
\item[\hskip \labelsep {\bfseries #1}\hskip \labelsep {\bfseries #2.}]}{\end{trivlist}}


\begin{document}

% --------------------------------------------------------------
%                         Start here
% --------------------------------------------------------------

%\renewcommand{\qedsymbol}{\filledbox}

\title{CS161 Week3 Project Plan}%replace X with the appropriate number
\author{Ziwei Wu\\ %replace with your name
\\
Student Number: 933296824} %if necessary, replace with your course title

\maketitle

%note * make the section un-numbered.

%Note 1: The * tells LaTeX not to number the lines.  If you remove the *, be sure to remove it below, too.
%Note 2: Inside the align environment, you do not want to use $-signs.  The reason for this is that this is already a math environment. This is why we have to include \text{} around any text inside the align environment.
\begin{exercise}{3A}\\
Write a program that asks the user how many integers they would like to enter.
You can assume they will enter a number $>= 1$.
The program will then prompt the user to enter that many integers.
After all the numbers have been entered,
the program should display the largest and smallest of those numbers.\\
\begin{table}[H]
\centering
\caption*{Testing Cases}
\label{my-label}
\begin{tabular}{|c|c|}
\hline
Input & Output  \\
\hline
1, 2, 3, 4  &  min:1, max:4 \\
\hline
999, 1, -1223231231, 999, -122323123 & min: -122323123, max: 999 \\
\hline
5668, 5668, 5668 & min: 5668, max: 5668 \\
\hline
-1001, -1, 0, - 506 & min: -1001, max: 0\\
\hline
0 & min: 0, max: 0\\
\hline
\end{tabular}
\end{table}
\noindent
\textbf{Pseudocode:}\\
Ask the user how many integers he/she wants to enter.\\
Initialize an variable called $InputNum$ to store this value.\\
\\
Initialize two variables called $minNum$ and $maxNum$.\\
Initialize a variable called $num$.\\
\\
Ask the user to enter $InputNum$ number of integers.\\
Store the first input to $num$.\\
Set both $minNum$ and $maxNum$ to the value of $num$.\\

\begin{algorithmic}[H]
    \FOR{from zero to $InputNum -1 $}
    \STATE{Reinitialize $num$ to zero}
    \STATE{Store the input to $num$}
    \IF {$num < minNum$}
    \STATE {Set $minNum$ to $num$}

    \ELSIF {$num > maxNum$}
    \STATE {Set $maxNum$ to $num$}
    \ENDIF
    \ENDFOR
\end{algorithmic}
Display the variable $minNum$ and $maxNum$ to the user.
\end{exercise}

\begin{exercise}{3B}\\
Write a program that prompts the user for the name of a file and then tries to open it.
If the input file is there and can be opened,
the program should read the list of integers in the file,
which will have one integer per line.
The program will then add together all the integers in the file,
create an output file called sum.txt,
and write the sum to that file (just that number - no additional text).
\begin{table}[H]
\centering
\caption*{Testing Cases}
\label{my-label}
\begin{tabular}{|c|c|}
\hline
Input & Output  \\
\hline
file with list of integers: ``number.txt'' & Sum of integers is written to file: ``sum.txt" \\
\hline
file cannot be open & ``file can not be opened"  \\
\hline
file is not there & ``file is not there" \\
\hline
file does not contain integers & ``file does not contain integers" \\
\hline
\end{tabular}
\end{table}
\noindent
\textbf{Pseudocode:}\\
Ask the user to enter a file name.\\
Set a variable called $sum$ and initialize it to zero.\\
Read the file.\\
\begin{algorithmic}[H]
    \IF {file can not be opened}
    \STATE {Print a message to user and terminate the program}

    \ELSIF {file is not there}
    \STATE {Print a message to user and terminate the program}

    \ELSIF {file doesn't contain list of integers}
    \STATE {Print a message to user and terminate the program}

    \ELSE

    \FOR{Each line in the file until the end of file}
    \STATE{read the integer and add it to $sum$}
    \ENDFOR
    \ENDIF
\end{algorithmic}
Output the value of $sum$ to file named ``sum.txt"\\
\end{exercise}

\begin{exercise}{3C}\\
Write a program that prompts the user for an integer that the player
(maybe the user, maybe someone else) will try to guess.
If the player's guess is higher than the target number,
the program should display "too high"  If the user's guess is lower than the target number,
the program should display "too low"  The program should use a loop that repeats
until the user correctly guesses the number.
Then the program should print how many guesses it took.\\
\begin{table}[H]
\centering
\caption*{Testing Cases}
\label{my-label}
\begin{tabular}{|c|c|}
\hline
Input & Output  \\
\hline
1 & Too Low - try again \\
\hline
10000 & Too High - try again \\
\hline
5000& Too High - try again \\
\hline
2500 & You guessed it right in 2 tries \\
\hline
\end{tabular}
\end{table}
\noindent
\textbf{Pseudocode:}\\
Initialize a variable called $answer$ and set it to zero.\\
Ask the user to enter a number as the answer.\\
Assign the input number to variable $answer$.\\
\\
Initialize a variable called $guess$ and set it to zero to store the guessed number.\\
Initialize a variable called $count$ and set it to zero to store the number of
times that player has guessed. \\
\\
Ask the player to enter a number to make a guess and set $guess$ to that input value.\\
Increment the $count$ by 1.\\
\begin{algorithmic}[H]
    \WHILE {$guess$ does not equal to $answer$}
    \IF {$guess < answer$}
    \STATE{Tell the player the guess is too small
    \STATE{Ask for another guess and assign to $guess$
    variable}
    \STATE {Increment the $count$ by 1}
    \STATE{Jump back to the beginning of while loop}

    \ELSIF {$guess > answer$}
    \STATE{Tell the player the guess is too large
    \STATE{Ask for another guess and assign to $guess$
    variable}
    \STATE {Increment the $count$ by 1}
    \STATE{Jump back to the beginning of while loop}
    \ENDIF
    \ENDWHILE
\end{algorithmic}
Tells the player that he/she has made the right guessed.\\
Print the $count$ to tell the player how many times he/she guessed.

\end{exercise}


% --------------------------------------------------------------
%     You don't have to mess with anything below this line.
% --------------------------------------------------------------

\end{document}
