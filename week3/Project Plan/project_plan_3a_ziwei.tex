% --------------------------------------------------------------
% This is all preamble stuff that you don't have to worry about.
% Head down to where it says "Start here"
% --------------------------------------------------------------

\documentclass[11pt]{article}

\usepackage[margin=1in]{geometry}
\usepackage{float,amsmath,amsthm,amssymb,caption}
\usepackage{algorithmic, algorithm}
\newcommand{\N}{\mathbb{N}}
\newcommand{\Z}{\mathbb{Z}}

\newenvironment{theorem}[2][Theorem]{\begin{trivlist}
\item[\hskip \labelsep {\bfseries #1}\hskip \labelsep {\bfseries #2.}]}{\end{trivlist}}
\newenvironment{lemma}[2][Lemma]{\begin{trivlist}
\item[\hskip \labelsep {\bfseries #1}\hskip \labelsep {\bfseries #2.}]}{\end{trivlist}}
\newenvironment{exercise}[2]{\begin{trivlist}
\item[\hskip \labelsep {\bfseries #1}\hskip \labelsep {\bfseries #2.}]}{\end{trivlist}}
\newenvironment{reflection}[2][Reflection]{\begin{trivlist}
\item[\hskip \labelsep {\bfseries #1}\hskip \labelsep {\bfseries #2.}]}{\end{trivlist}}
\newenvironment{proposition}[2][Proposition]{\begin{trivlist}
\item[\hskip \labelsep {\bfseries #1}\hskip \labelsep {\bfseries #2.}]}{\end{trivlist}}
\newenvironment{corollary}[2][Corollary]{\begin{trivlist}
\item[\hskip \labelsep {\bfseries #1}\hskip \labelsep {\bfseries #2.}]}{\end{trivlist}}


\begin{document}

% --------------------------------------------------------------
%                         Start here
% --------------------------------------------------------------

%\renewcommand{\qedsymbol}{\filledbox}

\title{CS161 Week3 Project Plan Part a}%replace X with the appropriate number
\author{Ziwei Wu} %replace with your name
 %if necessary, replace with your course title

\maketitle

%note * make the section un-numbered.

%Note 1: The * tells LaTeX not to number the lines.  If you remove the *, be sure to remove it below, too.
%Note 2: Inside the align environment, you do not want to use $-signs.  The reason for this is that this is already a math environment. This is why we have to include \text{} around any text inside the align environment.
\begin{exercise}{3A}\\
Write a program that asks the user how many integers they would like to enter.
You can assume they will enter a number $>= 1$.
The program will then prompt the user to enter that many integers.
After all the numbers have been entered,
the program should display the largest and smallest of those numbers.\\
\begin{table}[H]
\centering
\caption*{Testing Cases}
\label{my-label}
\begin{tabular}{|c|c|}
\hline
Input & Output  \\
\hline
1, 2, 3, 4  &  min:1, max:4 \\
\hline
999, 1, -1223231231, 999, -122323123 & min: -122323123, max: 999 \\
\hline
5668, 5668, 5668 & min: 5668, max: 5668 \\
\hline
-1001, -1, 0, - 506 & min: -1001, max: 0\\
\hline
0 & min: 0, max: 0\\
\hline
\end{tabular}
\end{table}
\noindent
\textbf{Pseudocode:}\\
Ask the user how many integers he/she wants to enter.\\
Initialize an variable called $InputNum$ to store this value.\\
\\
Initialize two variables called $minNum$ and $maxNum$.\\
Initialize a variable called $num$.\\
\\
Ask the user to enter $InputNum$ number of integers.\\
Store the first input to $num$.\\
Set both $minNum$ and $maxNum$ to the value of $num$.\\

\begin{algorithmic}[H]
    \FOR{from zero to $InputNum -1 $}
    \STATE{Reinitialize $num$ to zero}
    \STATE{Store the input to $num$}
    \IF {$num < minNum$}
    \STATE {Set $minNum$ to $num$}

    \ELSIF {$num > maxNum$}
    \STATE {Set $maxNum$ to $num$}
    \ENDIF
    \ENDFOR
\end{algorithmic}
Display the variable $minNum$ and $maxNum$ to the user.
\end{exercise}

\end{document}
