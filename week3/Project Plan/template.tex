% --------------------------------------------------------------
% This is all preamble stuff that you don't have to worry about.
% Head down to where it says "Start here"
% --------------------------------------------------------------

\documentclass[11pt]{article}

\usepackage[margin=1in]{geometry}
\usepackage{amsmath,amsthm,amssymb}

\newcommand{\N}{\mathbb{N}}
\newcommand{\Z}{\mathbb{Z}}

\newenvironment{theorem}[2][Theorem]{\begin{trivlist}
\item[\hskip \labelsep {\bfseries #1}\hskip \labelsep {\bfseries #2.}]}{\end{trivlist}}
\newenvironment{lemma}[2][Lemma]{\begin{trivlist}
\item[\hskip \labelsep {\bfseries #1}\hskip \labelsep {\bfseries #2.}]}{\end{trivlist}}
\newenvironment{exercise}[2][Exercise]{\begin{trivlist}
\item[\hskip \labelsep {\bfseries #1}\hskip \labelsep {\bfseries #2.}]}{\end{trivlist}}
\newenvironment{reflection}[2][Reflection]{\begin{trivlist}
\item[\hskip \labelsep {\bfseries #1}\hskip \labelsep {\bfseries #2.}]}{\end{trivlist}}
\newenvironment{proposition}[2][Proposition]{\begin{trivlist}
\item[\hskip \labelsep {\bfseries #1}\hskip \labelsep {\bfseries #2.}]}{\end{trivlist}}
\newenvironment{corollary}[2][Corollary]{\begin{trivlist}
\item[\hskip \labelsep {\bfseries #1}\hskip \labelsep {\bfseries #2.}]}{\end{trivlist}}


\begin{document}

% --------------------------------------------------------------
%                         Start here
% --------------------------------------------------------------

%\renewcommand{\qedsymbol}{\filledbox}

\title{CS225 Week1 Assignment Part 3}%replace X with the appropriate number
\author{Ziwei Wu\\ %replace with your name
Student Number: 933296824} %if necessary, replace with your course title

\maketitle

%note * make the section un-numbered.
\section*{Section 3.1}
\begin{exercise}{18}
%Note 1: The * tells LaTeX not to number the lines.  If you remove the *, be sure to remove it below, too.
%Note 2: Inside the align environment, you do not want to use $-signs.  The reason for this is that this is already a math environment. This is why we have to include \text{} around any text inside the align environment.
Let D be the set of all students at your school,
and let M(s) be ``s is a math major", let C(s) be ``s is a
computer science student," and let E (s ) be ``s is an
engineering student." Express each of the following statements
using quantifiers, variables, and the predicates M(s),C(s), and E(s).\\
a. There is an engineering student who is a math major.\\
b. Every computer science student is an engineering students. \\
c. No computer science students are engineering students.\\
d. Some computer science students are also math majors.\\
e. Some computer science students are engineering students and some are not.\\
\\
\textbf{Solutions:}\\
a. s\ \in D | E(s) \wedge M(s)\\
b. \forall s \in D | C(s) \rightarrow E(s)\\
c. \forall s \in D | C(s) \rightarrow \sim E(s)\\
d. \exists s \in D | C(s) \wedge E(s), and\ \exists s \in D | C(s)\wedge \sim E(s)\\
\end{exercise}

\section*{Section 3.2}
\begin{exercise}{2}
Which of the following is a negation for ``All dogs are loyal"? More than one answer may be correct.\\
a. All dogs are disloyal.\\
b. No dogs are loyal.\\
c. Some dogs are disloyal.\\
d. Some dogs are loyal.\\
e. There is a disloyal animal that is not a dog. \\
f. There is a dog that is disloyal.\\
g. No animals that are not dogs are loyal.\\
h. Some animals that are not dogs are loyal.\\
\\
\textbf{Solution:}\\
let dog=x, domain=D, loyal=L\\
All dogs are loyal in formal language:\\
$$ \forall x \in D, L(x) $$\\
Its negation in formal language:\\
$$ \exists x \in D, \sim L(x)$$\\
Therefore, the choices best represents the negation is c and f.
\end{exercise}

\begin{exercise}{4}
Write an informal negation for each of the following statements.
Be careful to avoid negations that are ambiguous.\\
a. All dogs are friendly.\\
b. All people are happy.\\
c. Some suspicions were substantiated.\\
d. Some estimates are accurate. \\
\\
\textbf{Solutions:}\\
a. Some dogs are not friendly.\\
b. Some people are not happy.\\
c. All suspicions were unsubstantiated.\\
d. All estimates are inaccurate.\\

\end{exercise}

% --------------------------------------------------------------
%     You don't have to mess with anything below this line.
% --------------------------------------------------------------

\end{document}
